\section{Introdução}

O trabalho consiste na participação de uma competição do Kaggle. A competição prevê a predição de preços de casas baseada em suas características. Para isso, é necessário que seja aplicado o conhecimento adquirido no curso para fazer com que um programa de computador seja capaz de aprender como fazer essa predição baseado numa base de dados conhecida previamente.

Além de fornecer a base de dados de treino (que contém os preços das casas) e a base de teste (base usada para fazer a predição final), o Kaggle também fornece materiais de outros participantes, que auxiliam o processo de aprendizagem e desenvolvimento do programa.

Uma outra informação presente na competição é a descrição de cada uma das características possíveis das casas, para que auxilie o desenvolvedor a escolher o que fazer com cada uma das colunas da base de dados.

\subsection{Arquivos fornecidos pelo Kaggle}

\begin{itemize}
\item \textbf{$train.csv$} - dataset de treino
\item \textbf{$test.csv$} - dataset de teste
\item \textbf{$data_description.txt$} - descrição de cada uma das características da casa
\item \textbf{$sample_submission.csv$} - exemplo de arquivo para ser submetido ao Kaggle
\end{itemize}

\subsection{Como funciona a submissão à competição}

Para submeter a reposta ao Kaggle, o competidor deve criar um arquivo $ .csv $ (valores separados por vírgulas) contendo duas colunas: Id e SalePrice. Além disso, os preços e os Ids devem corresponder, para que o sistema do Kaggle consiga validar o preço para o seu respectivo Id.

Ao enviar o arquivo para a validação, o Kaggle verifica a pontuação da tentativa e informa a colocação.

\subsection{Premissas}

O trabalho apresenta, como premissa, o fato de se usar apenas algoritmos de regressão para realizar a predição dos preços. Além disso, deixou livre a escolha dos algoritmos por parte do participante. Sendo assim, os algoritmos escolhidos foram: Ridge, Lasso, ElasticNet e KernelRidge.

\clearpage